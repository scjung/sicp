\chapter{\Large{Building Abstractions with Procedures}}

%%%%%%%%%%%%%%%%%%%%%%%%%%%%%%%%%%%%%%%%%%%%%%%%%%%%%%%%%%%%%%%%%%%%%%
\exercise{1.1}
\begin{scheme}
> 10
10
\end{scheme}
\begin{scheme}
> (+ 5 3 4)
12
\end{scheme}
\begin{scheme}
> (- 9 1)
8
\end{scheme}
\begin{scheme}
> (/ 6 2)
3
\end{scheme}
\begin{scheme}
> (+ (* 2 4) (- 4 6))
> (+ 8 -2)
6
\end{scheme}
\begin{scheme}
> (define a 3)
> (define b (+ a 1))
\end{scheme}
\begin{scheme}
> (+ a b (* a b))
> (+ 3 4 (* 3 4))
> (+ 3 4 12)
19
\end{scheme}
\begin{scheme}
> (= a b)
> (= 3 4)
#f
\end{scheme}
\begin{scheme}
> (if (and (> b a) (< b (* a b))) b a)
> (if (and (> 4 3) (< 4 (* 3 4))) 4 3)
> (if (and #t (< 4 12)) 4 3)
> (if (and #t #t) 4 3)
4
\end{scheme}
\begin{scheme}
> (cond ((= a 4) 6)
        ((= b 4) (+ 6 7 a))
        (else 25))
> (cond ((= 3 4) 6)
        ((= 4 4) (+ 6 7 3))
        (else 25))
> (+ 6 7 3)
16
\end{scheme}
\begin{scheme}
> (+ 2 (if (> b a) b a))
> (+ 2 (if (> 4 3) 4 3))
> (+ 2 4)
6
\end{scheme}
\begin{scheme}
> (* (cond ((> a b) a)
           ((< a b) b)
           (else -1))
     (+ a 1))
> (* (cond ((> 3 4) 3)
           ((< 3 4) 4)
           (else -1))
     (+ 3 1))
> (* 4 4)
16
\end{scheme}

%%%%%%%%%%%%%%%%%%%%%%%%%%%%%%%%%%%%%%%%%%%%%%%%%%%%%%%%%%%%%%%%%%%%%%
\exercise{1.2}
\begin{scheme}
(/ (+ 5 4
      (- 2
         (- 3
            (+ 6 (/ 4 5)))))
   (* 3
      (- 6 2)
      (- 2 7)))
\end{scheme}

%%%%%%%%%%%%%%%%%%%%%%%%%%%%%%%%%%%%%%%%%%%%%%%%%%%%%%%%%%%%%%%%%%%%%%
\exercise{1.3}
\schemein{./src/ch01-ex03.ss}

%%%%%%%%%%%%%%%%%%%%%%%%%%%%%%%%%%%%%%%%%%%%%%%%%%%%%%%%%%%%%%%%%%%%%%
\exercise{1.4}
두번째 인자가 양수일때는 연산자가 \texttt{+}가 되므로 두 인자를 더하며,
음수일때는 연산자가 \texttt{-}가 되므로 두 인자를 뺀다.

%%%%%%%%%%%%%%%%%%%%%%%%%%%%%%%%%%%%%%%%%%%%%%%%%%%%%%%%%%%%%%%%%%%%%%
\exercise{1.5}
우선 applicative-order evaluation의 경우에는 다음과 같이 계산된다.
\begin{scheme}
> (test 0 (p))
> (test 0 (p))
> (test 0 (p))
  ...
\end{scheme}
이 경우에는 프로시저 \texttt{test}를 펼치기에 앞서 주어진 인자를 먼저 계산하여야
한다. 그런데 두번째 인자 \texttt{p}의 정의에 따르면 계산을 위해 아무리
치환하여도 다시 자기 자신으로 바뀌어 버리므로 치환은 영원히 끝나지 않는다. 즉
\texttt{(test 0 (p))}을 applicative-order evaluation 방식으로 계산하는 것은
불가능하다.\\

반면에 normal-order evaluation의 경우에는 다음과 같이 계산된다.
\begin{scheme}
> (test 0 (p))
> (if (= 0 0)
      0
      (p))
0
\end{scheme}
\texttt{p}에 대한 계산은 필요치 않으므로 무시되며, 따라서 \texttt{(test 0
  (p))}는 정상적으로 계산된다.

%%%%%%%%%%%%%%%%%%%%%%%%%%%%%%%%%%%%%%%%%%%%%%%%%%%%%%%%%%%%%%%%%%%%%%
\exercise{1.6}

\SCHEME의 프로시저 계산은 applicative-order evaluation 방식으로 수행되므로,
\texttt{new-if}를 풀어내기 전 먼저 인자로 들어오는 \texttt{predicate},
\texttt{then-clause}, \texttt{else-clause}를 계산하여야 한다. 그런데 새로
작성된 \texttt{sqrt-iter}에서 \texttt{new-if}에 주어지는
\texttt{else-clause}를 계산하려면 \texttt{sqrt-iter}를 계산해야 한다. 즉,
\texttt{sqrt-iter}를 계산하기 위해서 \texttt{sqrt-iter}를 계산해야 하므로
안자가 끝없이 풀리면서 프로시저 계산이 끝나지 않게 된다.\\

반면에 기존의 \texttt{if}는 \texttt{predicate}을 먼저 계산하고 난 후에
\texttt{then-clause}, \texttt{else-clause} 중 해당하는 표현식을 계산하므로
이러한 문제가 발생하지 않는다.

%%%%%%%%%%%%%%%%%%%%%%%%%%%%%%%%%%%%%%%%%%%%%%%%%%%%%%%%%%%%%%%%%%%%%%
\exercise{1.7}

교재의 프로그램으로는 다음과 같이 아주 작은 수의 제곱근을 정확히 찾지
못한다. 이는 \texttt{good-enough?}에서 쓰인 허용치로 얻을 수 있는 정확도가
구하려는 제곱근에 못 미치기 때문이다.

\begin{scheme}
> (sqrt (square 0.001))
0.03126065525445276
\end{scheme}

반대로 아주 큰 수에 대해서는 실수값 정확도의 한계로 인해 몫과 평균값이 정확히
구해지지 않아, 그 차이가 \texttt{good-enough?}에 쓰인 허용치 밖을 맴도는
현상이 발생할 수 있다. 이런 경우에는 계산이 끝나지 않게 된다. 예를 들어 다음과
같은 표현식은 대부분의 컴퓨터에서 계산해 낼 수 없다.

\begin{scheme}
> (sqrt 1e13)
\end{scheme}

다음은 이러한 문제를 해결하기 위해 교재에서 제안한 방식대로 구현한 코드이다
(이전과 동일한 프로시저는 생략함).

\schemein{./src/ch01-ex07.ss}

9,10번째 줄에 새로 정의된 \texttt{good-enough?} 프로시저를 볼 수 있다. 이
프로시저에서 쓰이는 인자 \texttt{old-guess}는 이전에 찾아낸 제곱근을, \texttt{guess}는
\texttt{old-guess}를 바탕으로 좀 더 정확히 구한 제곱근을 의미한다. 이 둘
사이의 차가 특정 수 이하가 되면 (즉, \texttt{improve} 프로시저가 만들어낸
차이가 미미할 경우) \texttt{sqrt-iter}는 계산을 마치게 된다. 19,20번째 줄의
\texttt{sqrt} 역시 변경되었는데, 이는 맨 처음 단계에서 항상 \texttt{improve}를
수행하도록 하기 위함이다.\\

다음은 새로 작성한 프로그램으로 아주 작은 수와 큰 수의 제곱근을 구한
것이다. 이전과 달리 문제없이 제곱근을 구할 수 있다.
\begin{scheme}
> (sqrt (square 0.001))
0.0010000001533016628
> (sqrt 1e13)
3162277.6601683795
\end{scheme}

%%%%%%%%%%%%%%%%%%%%%%%%%%%%%%%%%%%%%%%%%%%%%%%%%%%%%%%%%%%%%%%%%%%%%%
\exercise{1.8}

앞에서 작성한 제곱근 프로그램에서 \texttt{improve} 프로시저만 제시된 식처럼
수정하면 된다.

\schemein{./src/ch01-ex08.ss}

다음은 작성한 프로그램으로 여러 세제곱근을 구한 것이다.
\begin{scheme}
> (sqrt 8)
2.000000000012062
> (sqrt (* 0.001 0.001 0.001))
0.0010000009808132022
> (sqrt (* 1e13 1e13 1e13))
10000000000000.0
\end{scheme}

%%%%%%%%%%%%%%%%%%%%%%%%%%%%%%%%%%%%%%%%%%%%%%%%%%%%%%%%%%%%%%%%%%%%%%
\exercise{1.9}

첫번째 방식으로 구현되었을때 \texttt{(+ 4 5)}가 계산되는 과정은 다음과 같다.

\begin{scheme}
> (+ 4 5)
> (if (= 4 0) 5 (inc (+ (dec 4) 5)))
> (inc (+ 3 5))
> (inc (if (= 3 0) 5 (inc (+ (dec 3) 5))))
> (inc (inc (+ 2 5)))
> (inc (inc (if (= 2 0) 5 (inc (+ (dec 2) 5)))))
> (inc (inc (inc (+ 1 5))))
> (inc (inc (inc (if (= 1 0) 5 (inc (+ (dec 1) 5))))))
> (inc (inc (inc (inc (+ 0 5)))))
> (inc (inc (inc (inc (if (= 0 0) 5 (inc (+ (dec 0) 5)))))))
> (inc (inc (inc (inc 5))))
> (inc (inc (inc 6)))
> (inc (inc 7))
> (inc 8)
9
\end{scheme}

즉 첫번째 인자에 선형인 재귀 프로세스이다.\\

다음은 두번째 방식으로 구현되었을때 \texttt{(+ 4 5)}가 계산되는 과정이다.

\begin{scheme}
> (+ 4 5)
> (if (= 4 0) 5 (+ (dec 4) (inc 5)))
> (+ 3 6)
> (if (= 3 0) 6 (+ (dec 3) (inc 6)))
> (+ 2 7)
> (if (= 2 0) 7 (+ (dec 2) (inc 7)))
> (+ 1 8)
> (if (= 1 0) 8 (+ (dec 1) (inc 8)))
> (+ 0 9)
> (if (= 0 0) 9 (+ (dec 0) (inc 9)))
9
\end{scheme}

앞의 구현과 달리 여기서는 첫번째 인자에 선형으로 반복 프로세스를
수행한다. 이는 교재의 \texttt{fact-iter}와 유사한 구현이다. 주어지는 첫번째
인자는 \texttt{fact-iter}의 \texttt{counter}와 같이 반복이 얼마나 더
수행되어야 하는지를 나타낸다. 그리고 두번째 인자는 \texttt{fact-iter}의
\texttt{product}와 같이 이제까지 구해낸 부분합을 나타낸다.

%%%%%%%%%%%%%%%%%%%%%%%%%%%%%%%%%%%%%%%%%%%%%%%%%%%%%%%%%%%%%%%%%%%%%%
\exercise{1.10}

이하는 주어진 표현식이 계산되는 과정을 나타낸 것이다. 간략하게 표현하기 위해
프로시저 몸체로 치환되는 과정은 생략하였다.

\begin{scheme}
> (A 1 10)
> (A 0 (A 1 9))
> (A 0 (A 0 (A 1 8)))
> (A 0 (A 0 (A 0 (A 1 7))))
> (A 0 (A 0 (A 0 (A 0 (A 1 6)))))
> (A 0 (A 0 (A 0 (A 0 (A 0 (A 1 5))))))
> (A 0 (A 0 (A 0 (A 0 (A 0 (A 0 (A 1 4)))))))
> (A 0 (A 0 (A 0 (A 0 (A 0 (A 0 (A 0 (A 1 3))))))))
> (A 0 (A 0 (A 0 (A 0 (A 0 (A 0 (A 0 (A 0 (A 1 2)))))))))
> (A 0 (A 0 (A 0 (A 0 (A 0 (A 0 (A 0 (A 0 (A 0 (A 1 1))))))))))
> (A 0 (A 0 (A 0 (A 0 (A 0 (A 0 (A 0 (A 0 (A 0 2)))))))))
> (A 0 (A 0 (A 0 (A 0 (A 0 (A 0 (A 0 (A 0 4))))))))
> (A 0 (A 0 (A 0 (A 0 (A 0 (A 0 (A 0 8)))))))
> (A 0 (A 0 (A 0 (A 0 (A 0 (A 0 16))))))
> (A 0 (A 0 (A 0 (A 0 (A 0 32)))))
> (A 0 (A 0 (A 0 (A 0 64))))
> (A 0 (A 0 (A 0 128)))
> (A 0 (A 0 256))
> (A 0 512)
1024
\end{scheme}

\begin{scheme}
> (A 2 4)
> (A 1 (A 2 3))
> (A 1 (A 1 (A 2 2)))
> (A 1 (A 1 (A 1 (A 2 1))))
> (A 1 (A 1 (A 1 2)))
> (A 1 (A 1 (A 0 (A 1 1))))
> (A 1 (A 1 (A 0 2)))
> (A 1 (A 1 4))
> (A 1 (A 0 (A 1 3)))
> (A 1 (A 0 (A 0 (A 1 2))))
> (A 1 (A 0 (A 0 (A 0 (A 1 1)))))
> (A 1 (A 0 (A 0 (A 0 2))))
> (A 1 (A 0 (A 0 4)))
> (A 1 (A 0 8))
> (A 1 16)
  ...
65536
\end{scheme}

\begin{scheme}
> (A 3 3)
  ...
65536
\end{scheme}

프로시저 \texttt{f}는 \texttt{A}에 첫번째
인자로 \texttt{0}을 주므로 항상 \texttt{A} 내 \texttt{cond}의 두번째 경우로
계산된다. 따라서 \texttt{(f n)}은 $2n$을 계산하는 것과 동일하다.\\

프로시저 \texttt{g}에 임의의 수를 넣어보면 다음과 같은 결과가 나온다.

\begin{scheme}
> (g 0)
0
> (g 1)
2
> (g 2)
> (A 0 (A 1 1))
> (A 0 2)
4
> (g 3)
> (A 0 (A 1 2))
> (A 0 (A 0 (A 1 1)))
> (A 0 (A 0 2))
> (A 0 2)
8
> (g 4)
16
  ...
\end{scheme}

실제로 프로시저 \texttt{g}가 계산되는 과정을 자세히 살펴보면, 우선 주어진 인자만큼
식 바깥쪽에 \texttt{(A 0 ...)}이 붙고, 맨 안쪽 식의 두번째 인자는 계속
줄어든다. 그러다가 맨 안쪽 식이 \texttt{(A 1 1)}의 형태를 이루면, 이것은
\texttt{A}의 정의에 따라 \texttt{2}로 치환된다. 이 다음부터는 \texttt{(A 0
  ...)}의 계산방식에 따라 맨 안쪽 값에 \texttt{2}가 계속 곱해지면서 식이
줄어들게 된다. 결론적으로, \texttt{(g~n)}은 $2^n$을 계산하는 것과 동일하다.\\

프로시저 \texttt{h}에 임의의 수를 넣어보면 다음과 같은 결과가 나온다.

\begin{scheme}
> (h 0)
1
> (h 1)
2
> (h 2)
> (A 1 (A 2 1))
> (g (h 1))        ; (A 1 n) = (g n), (A 2 n) = (h n)
4
> (h 3)
> (g (h 2))
> (g (g (h 1)))
> (g (g 2))
16
> (h 4)
> (g (g (g (h 1))))
> (g (g (g 2)))
65536
\end{scheme}

즉, \texttt{(h~n)}은 $g^{(n-1)}(2)$를 계산하는 것과 동일하다.\\

참고로 프로시저 \texttt{f}, \texttt{g}, \texttt{h} 모두 인자로 음수가 주어지면
값을 제대로 계산하지 못한다 (계산이 끝나지 않음). 따라서 엄밀히 말하면 여기서
애기한 수학적 함수를 제대로 구현한 것이라고 볼 수 없다.

%%%%%%%%%%%%%%%%%%%%%%%%%%%%%%%%%%%%%%%%%%%%%%%%%%%%%%%%%%%%%%%%%%%%%%
\exercise{1.11}

다음은 주어진 함수를 재귀 프로세스 및 반복 프로세스 방식으로 구현한 것이다.

\schemein{./src/ch01-ex11.ss}

반복 프로세스 방식의 구현에서 정의된 프로시저 \texttt{f-iter}에는 계산에
필요한 이전 계산값 $f(n-1)$, $f(n-2)$,$f(n-3)$이 인자로 넘어간다.

%%%%%%%%%%%%%%%%%%%%%%%%%%%%%%%%%%%%%%%%%%%%%%%%%%%%%%%%%%%%%%%%%%%%%%
\exercise{1.12}

파스칼의 삼각형에서 $i$번째 행의 왼쪽에서 $j$번째 수를 구하는 함수를 $f(i,j)$라
하자. 이러한 함수는 수학적으로 다음과 같이 정의할 수 있다.

\begin{equation}\notag
  f(i,j) =
  \begin{cases}
    1                    & \text{if } j=1 \vee i=j\\
    f(i-1,j-1) + f(i-1,j) & \text{otherwise}
  \end{cases}
\end{equation}
\begin{center}
  \qquad\qquad\qquad 단, $i>1$이고 $1\le j \le i$일 경우에만 정의됨
\end{center}

이를 \SCHEME으로 구현한 코드는 다음과 같다 ($f$가 정의되는 범위내 인자만
사용된다고 가정함).

\schemein{./src/ch01-ex12.ss}

%%%%%%%%%%%%%%%%%%%%%%%%%%%%%%%%%%%%%%%%%%%%%%%%%%%%%%%%%%%%%%%%%%%%%%
\exercise{1.13}

교재에서 제시된 방식대로 $\phi=(1+\sqrt{5})/2$, $\psi=(1-\sqrt{5})/2$ 일 때,
$\text{Fib}(n)=(\phi^n - \psi^n) / \sqrt{5}$임을 다음과 같이 귀납적으로 증명한다.
\newcommand{\FIB}[0]{\text{Fib}}
\begin{itemize}
\item $n=0$ 일 경우:
  \begin{align}\notag
    \FIB(0) &= 0 \\\notag
    \frac{\phi^0 - \psi^0}{\sqrt{5}} &= 0
  \end{align}
\item $n=1$ 일 경우:
  \begin{align}\notag
    \FIB(1) &= 1 \\\notag
    \frac{\phi - \psi}{\sqrt{5}} &= \frac{1+\sqrt{5}-1+\sqrt{5}}{2\sqrt{5}}
    \notag\\
    &= 1 \notag
  \end{align}
\item 귀납 가정: $n=k$ 일 경우 다음이 성립한다고 가정한다.
  \begin{align}\notag
    \FIB(k) & = \frac{\phi^k - \psi^k}{\sqrt{5}}
  \end{align}
\item $n=k+1$ 일 경우:
  \begin{align}\notag
    \FIB(k+1) &= \FIB(k) + \FIB(k-1)
  \end{align}

  귀납 가정을 $n=k-1$ 일 경우 적용하면 다음과 같다.
  \begin{align}
    \FIB(k-1) &= \frac{\phi^{k-1} - \psi^{k-1}}{\sqrt{5}}
    \notag\\
    &= \frac{\phi^k/\phi - \psi^k/\psi}{\sqrt{5}}
    \notag\\
    &= \frac{\frac{2\phi^k}{1+\sqrt{5}} - \frac{2\psi^k}{1-\sqrt{5}}}{\sqrt{5}}
    \notag\\
    &= \frac{(1+\sqrt{5})\psi^k-(1-\sqrt{5})\phi^k}{2\sqrt{5}}
    \notag\\
    &= \frac{\psi^k-\phi^k}{2\sqrt{5}}+\frac{\sqrt{5}\psi^k+\sqrt{5}\phi^k}{2\sqrt{5}}
    \notag\\
    &= \frac{-\FIB(k)}{2}+\frac{\psi^k+\phi^k}{2}
    \tag{귀납 가정에 의해, $\star$}
  \end{align}

  $n=k+1$일 경우의 증명하고자 하는 식을 정리하면 다음과 같다.
  \begin{align}
    \frac{\phi^{k+1} - \psi^{k+1}}{\sqrt{5}}
    &= \frac{(1+\sqrt{5})\phi^k-(1-\sqrt{5})\psi^k}{2\sqrt{5}}
    \notag\\
    &= \frac{\phi^k -\psi^k + \sqrt{5}\phi^k + \sqrt{5}\psi^k}{2\sqrt{5}}
    \notag\\
    &= \frac{\phi^k -\psi^k}{2\sqrt{5}} + \frac{\sqrt{5}\phi^k + \sqrt{5}\psi^k}{2\sqrt{5}}
    \notag\\
    &= \frac{\FIB(k)}{2} + \frac{\phi^k + \psi^k}{2}
    \tag{귀납 가정에 의해}\\
    &= \frac{\FIB(k)}{2} + \frac{\FIB(k)}{2} + \FIB(k-1)
    \tag{$\star$에 의해}\\
    &= \FIB(k) + \FIB(k-1)\notag
  \end{align}
  \QED
\end{itemize}

%%%%%%%%%%%%%%%%%%%%%%%%%%%%%%%%%%%%%%%%%%%%%%%%%%%%%%%%%%%%%%%%%%%%%%
\exercise{1.14}

그림 \ref{fig:count-change}\는 \texttt{(count-change 11)}가 계산되는 과정을 트리
형태로 나타낸 것이다.

\TODO

\begin{figure}[t]
  \centering
  \includegraphics[height=.65\textheight]{ch01-ex14}
  \caption{\texttt{(count-change 11)} 계산 과정을 나타내는 트리}
  \label{fig:count-change}
\end{figure}

%%%%%%%%%%%%%%%%%%%%%%%%%%%%%%%%%%%%%%%%%%%%%%%%%%%%%%%%%%%%%%%%%%%%%%
\exercise{1.15}

다음은 \texttt{(sine 12.15)}가 계산되는 도중 프로시저 \texttt{p}가 호출되는
모습을 나타낸 것이다 (몇몇 수는 소수점 이하가 정확하지 않을 수 있음).

\begin{scheme}
> (sine 12.15)
> (p (sine 4.05))
> (p (p (sine 1.3499999999)))
> (p (p (p (sine 0.4499999999))))
> (p (p (p (p (sine 0.15)))))
> (p (p (p (p (p (sine 0.049999999999))))))
> (p (p (p (p 0.1495))))
> (p (p (p 0.43513455))))
> (p (p 0.97584653))
> (p -0.7895631)
-0.39980345741334
\end{scheme}

\renewcommand{\theenumi}{\alph{enumi}}
\begin{enumerate}
\item \texttt{(sine 12.15)}를 계산할 때 프로시저 \texttt{p}는 5번 적용되었다.
\item 프로시저 \texttt{p}는 주어진 인자에 계속 3을 나눠가면서 0.1 이하가 될
  때까지 적용된다. 즉 $\Theta(\log_3n)$의 계산횟수 증가율을 보인다. 그리고
  계산횟수가 증가함에 따라 동일한 비율로 프로시저 \texttt{p} 호출 횟수가
  증가하므로, 공간 증가율 역시 $\Theta(\log_3n)$로 볼 수 있다.
\end{enumerate}

%%%%%%%%%%%%%%%%%%%%%%%%%%%%%%%%%%%%%%%%%%%%%%%%%%%%%%%%%%%%%%%%%%%%%%
\exercise{1.16}
\schemein{./src/ch01-ex16.ss}

다음은 이 구현이 실제로 반복 프로세스 방식을 따르는지를 알아보기 위해
\texttt{(fast-expt~3~4)}가 계산되는 과정을 나열한 것이다.

\begin{scheme}
> (fast-expt 3 4)
> (expt-iter (1 3 4))
> (expt-iter (1 9 2))
> (expt-iter (1 81 1))
> (expt-iter (81 81 0))
81
\end{scheme}

%%%%%%%%%%%%%%%%%%%%%%%%%%%%%%%%%%%%%%%%%%%%%%%%%%%%%%%%%%%%%%%%%%%%%%
\exercise{1.17}

\schemein{./src/ch01-ex17.ss}

이 구현에서 프로시저가 반복되는 횟수는 \texttt{b}에 비례한다. 그런데
프로시저가 반복될 때마다 \texttt{b}는 1이 감소하거나 반으로 줄어드므로, 반복
횟수의 증가율은 $\Theta(\log n)$이다.


%%%%%%%%%%%%%%%%%%%%%%%%%%%%%%%%%%%%%%%%%%%%%%%%%%%%%%%%%%%%%%%%%%%%%%
\exercise{1.18}

\schemein{./src/ch01-ex18.ss}

%%%%%%%%%%%%%%%%%%%%%%%%%%%%%%%%%%%%%%%%%%%%%%%%%%%%%%%%%%%%%%%%%%%%%%
\exercise{1.19}

$T_{pq}^2(a,b)$는 계산 결과는 다음과 같다.
\begin{align}
  a &\la (bp+aq)q + (bq+aq+ap)q + (bq+aq+ap)p \notag\\
  b &\la (bp+aq)p + (bq+aq+ap)q \notag
\end{align}

각각의 결과를 정리하면 다음과 같다.
\begin{align}
    & (bp+aq)q + (bq+aq+ap)q + (bq+aq+ap)p \notag\\
   =& b\cdot pq + a\cdot qq + b\cdot qq + a\cdot qq + a\cdot pq
      + b\cdot pq + a\cdot pq + a\cdot pp \notag\\
   =& b(2pq + qq) + a(2pq + qq) + a(pp+qq) \notag\\
   \notag\\
    & (bp+aq)p + (bq+aq+ap)q \notag\\
   =& b\cdot pp + a\cdot pq + b\cdot qq + a\cdot qq + a\cdot pq \notag\\
   =& b(pp+qq) + a(2pq+qq) \notag
\end{align}

즉, 다음과 같은 규칙이 성립한다.
\begin{equation}\notag
  T_{pq}^2 = T_{p'q'} \qquad \text{where } p'=pp+qq \text{~and~} q'=2pq+qq
\end{equation}

다음은 이러한 규칙을 이용하여 작성한 프로그램이다.
\schemein{./src/ch01-ex19.ss}

%%%%%%%%%%%%%%%%%%%%%%%%%%%%%%%%%%%%%%%%%%%%%%%%%%%%%%%%%%%%%%%%%%%%%%
\exercise{1.20}

다음은 normal-order 계산 방식을 사용할 때 \texttt{(gcd 206 40)}이 계산되는
과정을 나타낸 것이다. \texttt{remainder}는 \texttt{r}로 줄여서
기술하였다. 그리고 이전 단계에서 넘어오면서 \texttt{remainder}가 수행된 횟수를
오른쪽에 주석으로 기술하였다.

\begin{scheme}
> (gcd 206 40)
> (if (= 40 0) 206 (gcd 40 (r 206 40)))
> (gcd 40 (r 206 40))
> (if (= (r 206 40) 0)
    a
    (gcd (r 206 40) (r 40 (r 206 40))))
> (gcd (r 206 40) (r 40 (r 206 40)))                                 ; 1
> (if (= (r 40 (r 206 40)) 0)
      (r 206 40)
      (gcd (r 40 (r 206 40))
           (r (r 206 40) (r 40 (r 206 40)))))
> (gcd (r 40 (r 206 40)) (r (r 206 40) (r 40 (r 206 40))))           ; 2
> (if (= (r (r 206 40) (r 40 (r 206 40))) 0)
      (r 40 (r 206 40))
      (gcd (r (r 206 40) (r 40 (r 206 40)))
           (r (r 40 (r 206 40)) (r (r 206 40) (r 40 (r 206 40))))))
> (gcd (r (r 206 40) (r 40 (r 206 40)))
           (r (r 40 (r 206 40)) (r (r 206 40) (r 40 (r 206 40)))))   ; 4
> (if (= (r (r 40 (r 206 40)) (r (r 206 40) (r 40 (r 206 40)))) 0)
      (r (r 206 40) (r 40 (r 206 40)))
      (gcd (r (r 40 (r 206 40)) (r (r 206 40) (r 40 (r 206 40))))
           (r (r (r 206 40) (r 40 (r 206 40)))
              (r (r 40 (r 206 40)) (r (r 206 40) (r 40 (r 206 40)))))))
> (r (r 206 40) (r 40 (r 206 40)))                                   ; 7
> (r 6 4)                                                            ; 3
2                                                                    ; 1
\end{scheme}

앞의 과정에서 볼 수 있듯이, normal-order 방식에서는 \texttt{gcd}를 계산하는
과정에서 계산이 미뤄지는 \texttt{remainder}가 자꾸만 늘어나게 된다. 결국
\texttt{(gcd 206 40)}을 계산하기 위해 총 18번 \texttt{remainder}를 계산해야
한다. 이 중 대부분은 중복된 계산이다. \\

다음은 applicative-order 방식에서의 \texttt{(gcd 206 40)}
계산과정을 나타낸 것이다. 앞의 경우와 달리 4번만 \texttt{remainder}를
계산하면 된다.

\begin{scheme}
> (gcd 206 40)
> (if (= 40 0) 206 (gcd 40 (r 206 40)))
> (gcd 40 (r 206 40))
> (gcd 40 6)                            ; 1
> (if (= 6 0) 40 (gcd 6 (r 40 6)))
> (gcd 6 (r 40 6))
> (gcd 6 4)                             ; 1
> (if (= 4 0) 6 (gcd 4 (r 6 4)))
> (gcd 4 (r 6 4))
> (gcd 4 2)                             ; 1
> (if (= 2 0) 4 (gcd 2 (r 4 2)))
> (gcd 2 (r 4 2))
> (gcd 2 0)                             ; 1
> (if (= 0 0) 2 (gcd 0 (r 2 0))) 
2
\end{scheme}

%%%%%%%%%%%%%%%%%%%%%%%%%%%%%%%%%%%%%%%%%%%%%%%%%%%%%%%%%%%%%%%%%%%%%%
\exercise{1.21}

\begin{scheme}
> (smallest-divisor 199)
199
> (smallest-divisor 1999)
1999
> (smallest-divisor 19999)
7
\end{scheme}

%%%%%%%%%%%%%%%%%%%%%%%%%%%%%%%%%%%%%%%%%%%%%%%%%%%%%%%%%%%%%%%%%%%%%%
\exercise{1.22}
\schemein{./src/ch01-ex22.ss}

다음은 구현한 프로그램을 이용하여 1000; 10,000; 100,000; 1,000,000 이상의 수
중 가장 작은 소수 3개를 찾아본 것이다. 소수가 아닌 것에 대한 결과는 생략하였다.
\begin{scheme}
> (search-for-primes 1000 1020)
 ...
1009 *** 24
 ...
1013 *** 23
 ...
1019 *** 22
> (search-for-primes 10000 10040)
 ...
10007 *** 47
10009 *** 46
 ...
10037 *** 47
 ...
> (search-for-primes 100000 100050)
 ...
100003 *** 191
 ...
100019 *** 180
 ...
100043 *** 188
 ...
> (search-for-primes 1000000 1000040)
 ...
1000003 *** 570
 ...
1000033 *** 558
 ...
1000037 *** 555
 ...
\end{scheme}
$24 \times \sqrt{10} \simeq 75.8946$이므로 1000과 10,000에 대해서는 예측이 잘
맞지 않는다. 반면에 100,000과 1,000,000의 경우에는 $180 \times \sqrt{10}
\simeq 569.2100$이므로 예측이 어느정도 맞는다. 즉 적은 계산량이 요구될 때는
실제 눈에 보이는 계산 이외의 요소가 크게 작용하므로 예측에 잘 맞지 않으나, 큰
계산량이 요구되면 그러한 부가적인 요소의 차지하는 비중이 적어지므로 예측의 정확도가
높아지게 된다.

%%%%%%%%%%%%%%%%%%%%%%%%%%%%%%%%%%%%%%%%%%%%%%%%%%%%%%%%%%%%%%%%%%%%%%
\exercise{1.23}
\schemein{./src/ch01-ex23.ss}

새로 작성한 프로그램으로 앞의 소수를 다시 검사하면 다음과 같은 결과를 얻을 수
있다.
\begin{scheme}
> (timed-prime-test 1009)

1009 *** 32
> (timed-prime-test 10007)

10007 *** 50
> (timed-prime-test 100003)

100003 *** 130
> (timed-prime-test 1000003)

1000003 *** 377
\end{scheme}

예상과 달리 계산 속도가 두 배로 빨라지지 않는 것을 볼 수 있다. 앞의 두
경우에는 오히려 속도가 느려지고, 나머지 두 경우에는 두 배가 아닌 대략 1.5배
가량 빨라졌다. 이는 새로 추가된 \texttt{next} 프로시저를 계산하는 과정에서
발생하는 부가적 비용 때문인 것으로 보인다.

%%%%%%%%%%%%%%%%%%%%%%%%%%%%%%%%%%%%%%%%%%%%%%%%%%%%%%%%%%%%%%%%%%%%%%
\exercise{1.24}
\schemein{./src/ch01-ex24.ss}

새로 작성한 프로그램으로 앞의 소수를 다시 검사하면 다음과 같은 결과를 얻을 수
있다.
\begin{scheme}
> (timed-prime-test 1009)

1009 *** 1254
> (timed-prime-test 10007)

10007 *** 1410
> (timed-prime-test 100003)

100003 *** 2878
> (timed-prime-test 1000003)

1000003 *** 4008
\end{scheme}

$\frac{\log{1000003}}{\log{1009}} \simeq 2$이므로 1,000,003을 검사하는 것은
1,009를 검사하는 것보다 시간이 약 두 배 정도 더 걸린다고 예측할 수 있다. 하지만
앞의 결과에서 볼 수 있듯이 실제로는 두 배 보다 더 많은 시간이 걸린다. 이러한
결과를 보이는 이유는 \texttt{ferma-test} 프로시저에서 알고리즘과 직접적인
연관은 없지만 부가적으로 드는 비용 존재하기 때문이다 (예를 들어
\texttt{random} 프로시저 계산에 따른 비용).

%%%%%%%%%%%%%%%%%%%%%%%%%%%%%%%%%%%%%%%%%%%%%%%%%%%%%%%%%%%%%%%%%%%%%%
\exercise{1.25}

새로 구현된 \texttt{expmod} 프로시저가 내주는 결과는 이전 프로시저와
동일하다. 하지만 둘 사이에는 계산 방식에서 미묘한 차이가 있다. 이전
프로시저에서는 되도는 과정이 끝나고 이전 계산으로 돌아갈 때 나머지를
취하면서 (\texttt{cond}의 2, 3번째 경우 참고) 돌아간다. 반면에 새로 구현된
프로시저에서는 \texttt{fast-expt}의 되도는 과정이 완전히 끝나서 최종 결과를
얻은 후에 나머지를 취한다. 즉 이 경우에는 나머지가 취해지지 않은 큰 수에
대해 계속 연산을 수행해야 하므로 입력값이 커질수록 계산량이 급격히 늘어나게
된다. 따라서 빠른 소수 검사에 적용하기엔 적합하지 않다.

%%%%%%%%%%%%%%%%%%%%%%%%%%%%%%%%%%%%%%%%%%%%%%%%%%%%%%%%%%%%%%%%%%%%%%
\exercise{1.26}

\texttt{square}를 사용하는 경우에는, 인자먼저 계산법에 의해 \texttt{expmod}
프로시저가 계산된 결과를 두 번 곱한다. 하지만 제시된 코드와 같이
\texttt{expmod}를 두 번 적으면 두 프로시저 계산의 결과가 따로 계산된 후, 그 두
결과를 곱하므로 계산량이 두 배가 된다.

%%%%%%%%%%%%%%%%%%%%%%%%%%%%%%%%%%%%%%%%%%%%%%%%%%%%%%%%%%%%%%%%%%%%%%
\exercise{1.27}

다음 프로시저는 주어진 수가 Carmichael 수인지를 판별한다.

\schemeinp{./src/ch01-ex27.ss}{28}{33}

다음은 작성한 프로시저로 몇가지 Carmichael 수를 검사해 본 것이다.

\begin{scheme}
> (carmichael-num? 561)
#t
> (carmichael-num? 1105)
#t
> (carmichael-num? 1729)
#t
> (carmichael-num? 2465)
#t
> (carmichael-num? 2821)
#t
> (carmichael-num? 6601)
#t
\end{scheme}

%%%%%%%%%%%%%%%%%%%%%%%%%%%%%%%%%%%%%%%%%%%%%%%%%%%%%%%%%%%%%%%%%%%%%%
\exercise{1.28}
\schemein{./src/ch01-ex28.ss}

다음은 새로 작성한 프로그램으로 몇가지 수의 소수 여부를 검사해 본
결과이다\footnote{Miller-Robin 검사 또한 Fermat 검사와 같이 확률적
  알고리즘이므로 소수가 아닌 수이어도 때에 따라서 소수라고 말할 수도
  있다. 하지만 Fermat 검사와 달리, Carmichael 수에 대해서도 검사를
  반복하면 언젠가는 반드시 소수가 아님을 알아낼 수 있다.}. 앞에서와
달리 Carmichael 수가 소수가 아님을 알아낸다.

\begin{scheme}
> (timed-prime-test 2)

2 *** 2399
> (timed-prime-test 3)

3 *** 2989
> (timed-prime-test 4)

4
> (timed-prime-test 5)

5 *** 4396
> (timed-prime-test 6)

6
> (timed-prime-test 561)

561
> (timed-prime-test 1105)

1105
> (timed-prime-test 1729)

1729
> (timed-prime-test 2465)

2465
> (timed-prime-test 2821)

2821
> (timed-prime-test 6601)

6601
\end{scheme}

%%%%%%%%%%%%%%%%%%%%%%%%%%%%%%%%%%%%%%%%%%%%%%%%%%%%%%%%%%%%%%%%%%%%%%
\exercise{1.29}
\schemein{./src/ch01-ex29.ss}

다음 결과에서 볼 수 있듯이 Simpson's Rule을 사용한 알고리즘이 더 정확한 결과를
내 준다.

\begin{scheme}
> (simpson-integral cube 0 1 100)
#e0.25
> (simpson-integral cube 0 1 1000)
#e0.25
\end{scheme}

%%%%%%%%%%%%%%%%%%%%%%%%%%%%%%%%%%%%%%%%%%%%%%%%%%%%%%%%%%%%%%%%%%%%%%
\exercise{1.30}
\schemein{./src/ch01-ex30.ss}

%%%%%%%%%%%%%%%%%%%%%%%%%%%%%%%%%%%%%%%%%%%%%%%%%%%%%%%%%%%%%%%%%%%%%%
\exercise{1.31}

\begin{enumerate}
\item 다음은 \texttt{product} 프로시저와 이것으로 구현한 \texttt{factorial},
  $\pi$의 근사값을 구하는 \texttt{approx-pi}
  프로시저이다. \texttt{approx-pi}의 인자는 $\pi$를 구할 때 쓰이는 항의 갯수를
  나타낸다. 이 값이 클수록 더 정확한 결과를 얻을 수 있다.
  \schemeinp{./src/ch01-ex31.ss}{3}{17}

  다음은 작성한 프로시저로 $\pi$를 구해본 것이다.
  \begin{scheme}
> (approx-pi 10)
#e3.2751010413348075685738023...
> (approx-pi 100)
#e3.1570301764551677396727431...
> (approx-pi 200)
#e3.1493784731686008593667128...
  \end{scheme}
\item 앞에서 작성한 \texttt{product} 프로시저는 재귀 프로세스를 따른다. 다음은
  이를 반복 프로세스를 따르도록 다시 작성한 것이다.
  \schemeinp{./src/ch01-ex31.ss}{21}{26}
\end{enumerate}

%%%%%%%%%%%%%%%%%%%%%%%%%%%%%%%%%%%%%%%%%%%%%%%%%%%%%%%%%%%%%%%%%%%%%%
\exercise{1.32}
\begin{enumerate}
\item  다음은 \texttt{accumulate} 프로시저와 이것을 사용하여 다시 구현한
  \texttt{sum}, \texttt{product} 프로시저이다.
  \schemeinp{./src/ch01-ex32.ss}{3}{13}
\item 앞에서 작성한 \texttt{accumulate} 프로시저는 재귀 프로세스를
  따른다. 다음은 이를 반복 프로세스를 따르도록 다시 작성한 것이다.
  \schemeinp{./src/ch01-ex32.ss}{17}{22}
\end{enumerate}

%%%%%%%%%%%%%%%%%%%%%%%%%%%%%%%%%%%%%%%%%%%%%%%%%%%%%%%%%%%%%%%%%%%%%%
\exercise{1.33}
\schemeinp{./src/ch01-ex33.ss}{1}{9}
\begin{enumerate}
\item \ 
  \schemeinp{./src/ch01-ex33.ss}{34}{35}
\item \ 
  \schemeinp{./src/ch01-ex33.ss}{39}{41}
\end{enumerate}

%%%%%%%%%%%%%%%%%%%%%%%%%%%%%%%%%%%%%%%%%%%%%%%%%%%%%%%%%%%%%%%%%%%%%%
\exercise{1.34}

주어진 `\texttt{(f f)}'가 계산되는 과정을 나타내면 다음과 같다.

\begin{scheme}
> (f f)
> (f 2)   ; substitued with the body of f
> (2 2)   ; substitued with the body of f
\end{scheme}

즉, 프로시저 이름이 와야 할 자리에 `\texttt{2}'가 오게 되므로 오류가 발생한다.

%%%%%%%%%%%%%%%%%%%%%%%%%%%%%%%%%%%%%%%%%%%%%%%%%%%%%%%%%%%%%%%%%%%%%%
\exercise{1.35}
황금비 $x$를 구하는 공식은 다음과 같다.
\begin{equation}\notag
  x^2 = x + 1
\end{equation}

양변을 $x$로 나누면 다음과 같다.
\begin{equation}\notag
  x = 1 + \frac{1}{x}
\end{equation}

즉, $x \mapsto 1 + 1/x$의 고정점을 구하면 된다. 이는 다음과 같이
\texttt{fixed-point} 프로시저를 이용해 구할 수 있다.

\begin{scheme}
> (fixed-point (lambda (x) (+ 1 (/ 1 x))) 1.0)
1.6180327868852458
\end{scheme}

%%%%%%%%%%%%%%%%%%%%%%%%%%%%%%%%%%%%%%%%%%%%%%%%%%%%%%%%%%%%%%%%%%%%%%
\exercise{1.36}

다음은 고정점이 구해지는 과정을 출력하도록 변경한 프로그램이다.
\schemeinp{./src/ch01-ex36.ss}{1}{18}

다음은 주어진 방식대로 $x^x=1000$의 고정점을 구한 것이다.

\begin{scheme}
> (fixed-point (lambda (x) (/ (log 1000) (log x))) 10.0)
1: 10.0
2: 2.9999999999999996
3: 6.2877098228681545
4: 3.7570797902002955
5: 5.218748919675316
6: 4.1807977460633134
7: 4.828902657081293
8: 4.386936895811029
9: 4.671722808746095
10: 4.481109436117821
11: 4.605567315585735
12: 4.522955348093164
13: 4.577201597629606
14: 4.541325786357399
15: 4.564940905198754
16: 4.549347961475409
17: 4.5596228442307565
18: 4.552843114094703
19: 4.55731263660315
20: 4.554364381825887
21: 4.556308401465587
22: 4.555026226620339
23: 4.55587174038325
24: 4.555314115211184
25: 4.555681847896976
26: 4.555439330395129
27: 4.555599264136406
28: 4.555493789937456
29: 4.555563347820309
30: 4.555517475527901
31: 4.555547727376273
32: 4.555527776815261
33: 4.555540933824255
34: 4.555532257016376
4.555532257016376
\end{scheme}

앞에서 구한 고정점을 average damping을 적용하여 다시 구해보면 다음과 같다.
\begin{scheme}
> (fixed-point (lambda (x) (average x (/ (log 1000) (log x)))) 10.0)
1: 10.0
2: 6.5
3: 5.095215099176933
4: 4.668760681281611
5: 4.57585730576714
6: 4.559030116711325
7: 4.55613168520593
8: 4.555637206157649
9: 4.55555298754564
10: 4.555538647701617
11: 4.555536206185039
4.555536206185039
\end{scheme}

전자의 경우, 후자에 비해 상대적으로 단계마다 널뛰는 정도가 커서 고정점에
다다르는데 더 오랜 시간이 걸리는 것을 알 수 있다.

%%%%%%%%%%%%%%%%%%%%%%%%%%%%%%%%%%%%%%%%%%%%%%%%%%%%%%%%%%%%%%%%%%%%%%
\exercise{1.37}

\begin{enumerate}
\item 구현한 \texttt{cont-frac} 프로시저는 다음과 같다.
  \schemeinp{./src/ch01-ex37.ss}{1}{7}
  다음은 구현한 프로시저로 황금비를 구해본 것이다. $k \ge 11$이면 소수점
  4자리까지 정확한 값을 얻을 수 있다.
  \begin{scheme}
> (cont-frac (lambda (i) 1.0) (lambda (i) 1.0) 1)
1.0
> (cont-frac (lambda (i) 1.0) (lambda (i) 1.0) 2)
0.5
> (cont-frac (lambda (i) 1.0) (lambda (i) 1.0) 3)
0.6666666666666666
> (cont-frac (lambda (i) 1.0) (lambda (i) 1.0) 4)
0.6000000000000001
> (cont-frac (lambda (i) 1.0) (lambda (i) 1.0) 5)
0.625
> (cont-frac (lambda (i) 1.0) (lambda (i) 1.0) 10)
0.6179775280898876
> (cont-frac (lambda (i) 1.0) (lambda (i) 1.0) 11)
0.6180555555555556
  \end{scheme}
\item 앞에서 구현한 \texttt{cont-frac} 프로시저는 재귀 프로세스를
  수행한다. 이 프로시저는 $i$번째 항을 $i+1$번째 항의 결과를 얻고 난 후 계산하기 때문에,
  이대로는 반복 프로세스를 수행하도록 바꾸기가 그리 쉽지 않다.\footnote{재귀
    프로세스를 수행하는 프로시저를 반복 프로세스를 수행하도록 바꿀 때는 보통
    이전 시점까지 구해진 결과를 통해 이번 결과를 구하는 프로시저를 만든다.
    그런데 여기서는 프로시저로 넘길 결과($i$번째 항)를 구하기 위해
    다음 단계에서 구해질 값($i+1$번째 항) 을 미리 알아내야 하므로 보통의
    방식을 적용할 수가 없다.}
  따라서 반복 프로세스를 수행하도록 하기 위해서는 계산 순서를 거꾸로 하여
  제일 안쪽에서 바깥쪽 항을 구성해 나가는 방식을 취해야 한다. 즉, 맨
  처음에는 $k$번째 항을 구하고, 다음은 $k$번째 항을 바탕으로
  $k-1$번째 항을 구하는 식이다. 다음은 이러한 방식으로 구현한 프로시저이다.
  \schemeinp{./src/ch01-ex37.ss}{9}{14}
\end{enumerate}

%%%%%%%%%%%%%%%%%%%%%%%%%%%%%%%%%%%%%%%%%%%%%%%%%%%%%%%%%%%%%%%%%%%%%%
\exercise{1.38}
\schemeinp{./src/ch01-ex38.ss}{8}{16}

%%%%%%%%%%%%%%%%%%%%%%%%%%%%%%%%%%%%%%%%%%%%%%%%%%%%%%%%%%%%%%%%%%%%%%
\exercise{1.39}
\schemeinp{./src/ch01-ex39.ss}{10}{13}



%%% Local Variables: 
%%% mode: latex
%%% TeX-master: "master"
%%% End: 

\chapter{\Large{Building Abstractions with Procedures}}
\section{Exercise 1.1}
\begin{lisp}
> 10
10
\end{lisp}
\begin{lisp}
> (+ 5 3 4)
12
\end{lisp}
\begin{lisp}
> (- 9 1)
8
\end{lisp}
\begin{lisp}
> (/ 6 2)
3
\end{lisp}
\begin{lisp}
> (+ (* 2 4) (- 4 6))
> (+ 8 -2)
6
\end{lisp}
\begin{lisp}
> (define a 3)
> (define b (+ a 1))
\end{lisp}
\begin{lisp}
> (+ a b (* a b))
> (+ 3 4 (* 3 4))
> (+ 3 4 12)
19
\end{lisp}
\begin{lisp}
> (= a b)
> (= 3 4)
#f
\end{lisp}
\begin{lisp}
> (if (and (> b a) (< b (* a b))) b a)
> (if (and (> 4 3) (< 4 (* 3 4))) 4 3)
> (if (and #t (< 4 12)) 4 3)
> (if (and #t #t) 4 3)
4
\end{lisp}
\begin{lisp}
> (cond ((= a 4) 6)
        ((= b 4) (+ 6 7 a))
        (else 25))
> (cond ((= 3 4) 6)
        ((= 4 4) (+ 6 7 3))
        (else 25))
> (+ 6 7 3)
16
\end{lisp}
\begin{lisp}
> (+ 2 (if (> b a) b a))
> (+ 2 (if (> 4 3) 4 3))
> (+ 2 4)
6
\end{lisp}
\begin{lisp}
> (* (cond ((> a b) a)
           ((< a b) b)
           (else -1))
     (+ a 1))
> (* (cond ((> 3 4) 3)
           ((< 3 4) 4)
           (else -1))
     (+ 3 1))
> (* 4 4)
16
\end{lisp}

\section{Exercise 1.2}
\begin{lisp}
(/ (+ 5 4
      (- 2
         (- 3
            (+ 6 (/ 4 5)))))
   (* 3
      (- 6 2)
      (- 2 7)))
\end{lisp}

\section{Exercise 1.3}
\begin{lisp}
(define (square x) (* x x))
(define (sum-ex x y z)
  (if (> x y)
      (+ (square x) (square (if (> y z) y z)))
      (+ (square y) (square (if (> x z) x z)))))
\end{lisp}

\section{Exercise 1.4}
두번째 인자가 양수일때는 연산자가 \texttt{+}가 되므로 두 인자를 더하며,
음수일때는 연산자가 \texttt{-}가 되므로 두 인자를 뺀다.

\section{Exercise 1.5}
우선 applicative-order evaluation의 경우에는 다음과 같이 계산된다.
\begin{lisp}
> (test 0 (p))
> (test 0 (p))
> (test 0 (p))
  ...
\end{lisp}
이 경우에는 프로시저 \texttt{test}를 펼치기에 앞서 주어진 인자를 먼저 계산하여야
한다. 그런데 두번째 인자 \texttt{p}의 정의에 따르면 계산을 위해 아무리
치환하여도 다시 자기 자신으로 바뀌어 버리므로 치환은 영원히 끝나지 않는다. 즉
\texttt{(test 0 (p))}을 applicative-order evaluation 방식으로 계산하는 것은
불가능하다.\\

반면에 normal-order evaluation의 경우에는 다음과 같이 계산된다.
\begin{lisp}
> (test 0 (p))
> (if (= 0 0)
      0
      (p))
0
\end{lisp}
\texttt{p}에 대한 계산은 필요치 않으므로 무시되며, 따라서 \texttt{(test 0
  (p))}는 정상적으로 계산된다.

\section{Exercise 1.6}

\SCHEME의 프로시저 계산은 applicative-order evaluation 방식으로 수행되므로,
\texttt{new-if}를 풀어내기 전 먼저 인자로 들어오는 \texttt{predicate},
\texttt{then-clause}, \texttt{else-clause}를 계산하여야 한다. 그런데 새로
작성된 \texttt{sqrt-iter}에서 \texttt{new-if}에 주어지는
\texttt{else-clause}를 계산하려면 \texttt{sqrt-iter}를 계산해야 한다. 즉,
\texttt{sqrt-iter}를 계산하기 위해서 \texttt{sqrt-iter}를 계산해야 하므로
안자가 끝없이 풀리면서 프로시저 계산이 끝나지 않게 된다.\\

반면에 기존의 \texttt{if}는 \texttt{predicate}을 먼저 계산하고 난 후에
\texttt{then-clause}, \texttt{else-clause} 중 해당하는 표현식을 계산하므로
이러한 문제가 발생하지 않는다.

\section{Exercise 1.7}

다음과 같이, 아주 작은 수에 대해서는 정확한 제곱근을 찾지
못한다. \texttt{good-enough?}에 쓰인 허용치으로 얻을 수 있는 정확도가
구하려는 제곱근에 못 미치기 때문이다.

\begin{lisp}
> (sqrt (square 0.001))
0.03126065525445276
\end{lisp}

또한 아주 큰 수에 대해서는 실수값 정확도의 한계로 인해 몫과 평균값이 정확히
구해지지 않아, 그 차이가 \texttt{good-enough?}에 쓰인 허용치 밖을 맴도는
현상이 발생할 수 있다. 이런 경우에는 계산이 끝나지 않게 된다. 예를 들어 다음과
같은 표현식은 대부분의 컴퓨터에서 계산해 낼 수 없다.

\begin{lisp}
> (sqrt 1e13)
\end{lisp}



%%% Local Variables: 
%%% mode: latex
%%% TeX-master: "master"
%%% End: 

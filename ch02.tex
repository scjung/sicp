\chapter{\Large{Building Abstractions with Data}}

%%%%%%%%%%%%%%%%%%%%%%%%%%%%%%%%%%%%%%%%%%%%%%%%%%%%%%%%%%%%%%%%%%%%%%
\exercise{2.1}

문제에서 제시하는 정리 방법을 간단히 말하면 분모가 항상 양수가 되게 하라는
것이다. 즉 분모가 음수일 때 분자, 분모 각각의 부호를 바꾸어 주면 된다.

\scheme[6]{./src/ch02-ex01.ss}

%%%%%%%%%%%%%%%%%%%%%%%%%%%%%%%%%%%%%%%%%%%%%%%%%%%%%%%%%%%%%%%%%%%%%%
\exercise{2.2}

\scheme{./src/ch02-ex02.ss}

%%%%%%%%%%%%%%%%%%%%%%%%%%%%%%%%%%%%%%%%%%%%%%%%%%%%%%%%%%%%%%%%%%%%%%
\exercise{2.3}

다음 구현에서 나타난 \texttt{make-rect} 프로시저는 구성하고자 하는 직사각형의
왼쪽 상단과 오른쪽 하단의 점을 받고서
(모든 내각이 직각인 사실을 바탕으로) 모든 점을 구해낸 후 \KOEN{쌍}{pair}의 쌍으로
묶어낸다. 이러한 방식으로 구현된 전체 프로그램의 추상화 구조를 나타내면 그림
\ref{fig:absrect}과 같다. 그림에서 볼 수 있듯이, 다른 구조로 직사각형을
나타내고자 할 때에는 구성자와 꼭지점을 내주는 선택자만 변경하면 된다.
\begin{figure}[t]
  \centering
  \HR{2em}\fbox{\parbox{27em}{\centering%
      \texttt{width-rect height-rect perimeter-rect area-rect}%
    }}\HR{2em}\vspace{1em}\\
  네 변으로 구성되는 직사각형 \vspace{1em}\\
  \HR{4em}\fbox{\parbox{23em}{\centering%
      \texttt{top-segment-rect left-segment-rect ...}%
    }}\HR{4em}\vspace{1em}\\
  네 꼭지점으로 구성되는 직사각형 \vspace{1em}\\
  \HR{3em}\fbox{\parbox{25em}{\centering%
      \texttt{make-rect top-left-rect top-right-rect ...}}}\HR{3em}\vspace{1em}\\
  쌍의 쌍으로 구성되는 직사각형 \vspace{1em}\\
  \HR{11em}\fbox{\parbox{9em}{\centering\texttt{cons car cdr}}}\HR{11em}\vspace{1em}\\
  어떻게든 구현된 쌍\vspace{1em}\\
  \caption{직사각형 프로그램의 자료추상화 배리어}
  \label{fig:absrect}
\end{figure}

\scheme[29]{./src/ch02-ex03.ss}

%%%%%%%%%%%%%%%%%%%%%%%%%%%%%%%%%%%%%%%%%%%%%%%%%%%%%%%%%%%%%%%%%%%%%%
\exercise{2.4}

다음은 `\texttt{(car (cons x y))}'가 계산되는 과정을 나타낸 것이다.

\begin{lstlisting}[language=Scheme]
> (car (cons x y))
> (car (lambda (m) (m x y)))
> (lambda (m) (m x y)) (lambda (p q) p)
> (lambda (p q) p) x y
x
\end{lstlisting}

\texttt{cdr} 프로시저는 다음과 같이 구현할 수 있다.
\scheme[7]{./src/ch02-ex04.ss}

다음은 `\texttt{(cdr (cons x y))}'가 계산되는 과정을 나타낸 것이다.

\begin{lstlisting}[language=Scheme]
> (cdr (cons x y))
> (cdr (lambda (m) (m x y)))
> (lambda (m) (m x y)) (lambda (p q) q)
> (lambda (p q) q) x y
y
\end{lstlisting}


%%% Local Variables: 
%%% mode: latex
%%% TeX-master: "master"
%%% End: 

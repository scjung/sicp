\appendix
\chapter{Objective-Caml 구현}

이 장에서 기술된 코드는 앞에서 \SCHEME으로 작성된 코드를 \OCAML\footnote{\URL{http://caml.inria.fr}}로 다시 구현한
것이다.

%%%%%%%%%%%%%%%%%%%%%%%%%%%%%%%%%%%%%%%%%%%%%%%%%%%%%%%%%%%%%%%%%%%%%%
\exercise{1.3}
\ocamlin{./src/ch01-ex03.ml}

%%%%%%%%%%%%%%%%%%%%%%%%%%%%%%%%%%%%%%%%%%%%%%%%%%%%%%%%%%%%%%%%%%%%%%
\exercise{1.7}
\ocamlin{./src/ch01-ex07.ml}

%%%%%%%%%%%%%%%%%%%%%%%%%%%%%%%%%%%%%%%%%%%%%%%%%%%%%%%%%%%%%%%%%%%%%%
\exercise{1.8}
\ocamlin{./src/ch01-ex08.ml}

%%%%%%%%%%%%%%%%%%%%%%%%%%%%%%%%%%%%%%%%%%%%%%%%%%%%%%%%%%%%%%%%%%%%%%
\exercise{1.11}
\ocamlin{./src/ch01-ex11.ml}

%%%%%%%%%%%%%%%%%%%%%%%%%%%%%%%%%%%%%%%%%%%%%%%%%%%%%%%%%%%%%%%%%%%%%%
\exercise{1.12}
\ocamlin{./src/ch01-ex12.ml}

%%%%%%%%%%%%%%%%%%%%%%%%%%%%%%%%%%%%%%%%%%%%%%%%%%%%%%%%%%%%%%%%%%%%%%
\exercise{1.16}
\ocamlin{./src/ch01-ex16.ml}

%%%%%%%%%%%%%%%%%%%%%%%%%%%%%%%%%%%%%%%%%%%%%%%%%%%%%%%%%%%%%%%%%%%%%%
\exercise{1.17}
\ocamlin{./src/ch01-ex17.ml}

%%%%%%%%%%%%%%%%%%%%%%%%%%%%%%%%%%%%%%%%%%%%%%%%%%%%%%%%%%%%%%%%%%%%%%
\exercise{1.18}
\ocamlin{./src/ch01-ex18.ml}

%%%%%%%%%%%%%%%%%%%%%%%%%%%%%%%%%%%%%%%%%%%%%%%%%%%%%%%%%%%%%%%%%%%%%%
\exercise{1.19}
\ocamlin{./src/ch01-ex19.ml}

%%%%%%%%%%%%%%%%%%%%%%%%%%%%%%%%%%%%%%%%%%%%%%%%%%%%%%%%%%%%%%%%%%%%%%
\exercise{1.22}

\OCAML에서 \SCHEME의 \texttt{(runtime)}과 유사한 기능을 하는 함수는
\texttt{Unix} 라이브러리의 \texttt{Unix} 모듈에 있는 \texttt{gettimeofday}
함수이다. \texttt{Unix} 라이브러리는 표준 라이브러리가 아니므로 자동으로
로드되지 않는다. 따라서 사용자가 직접 \texttt{unix.cma} 혹은
\texttt{unix.cmxa}를 사용할 것임을 명시해야 한다\footnote{\OCAML~인터랙티브
  환경에서는 프롬프트에서 \texttt{\#load "unix.cma"} 명령을 수행하면
  \texttt{Unix} 모듈을 사용할 수 있다.}.

\ocamlin{./src/ch01-ex22.ml}

%%%%%%%%%%%%%%%%%%%%%%%%%%%%%%%%%%%%%%%%%%%%%%%%%%%%%%%%%%%%%%%%%%%%%%
\exercise{1.23}
\ocamlin{./src/ch01-ex23.ml}

%%%%%%%%%%%%%%%%%%%%%%%%%%%%%%%%%%%%%%%%%%%%%%%%%%%%%%%%%%%%%%%%%%%%%%
\exercise{1.24}

\OCAML의 \texttt{int} 타입은 큰 수를 다룰 수가 없다. 따라서 \texttt{expmod}
함수를 \texttt{int} 값에 대해 계산하도록 작성하면 \KOEN{정수넘침}{integer overflow}이
발생하여 잘못된 계산이 수행될 수 있다. 이러한 문제를 해결하기 위해 여기서는
일반적인 정수 계산 대신 \texttt{Num} 라이브러리의 \texttt{Big\_int} 모듈에서
제공하는 함수를 사용한다. \texttt{Num} 라이브러리 또한 \texttt{Unix}
라이브러리처럼 표준 라이브러리가 아니므로 사용자가 직접 \texttt{nums.cma} 혹은
\texttt{nums.cmxa}를 사용할 것임을 명시해야 한다.

\ocamlin{./src/ch01-ex24.ml}


%%% Local Variables: 
%%% mode: latex
%%% TeX-master: "master"
%%% End: 
